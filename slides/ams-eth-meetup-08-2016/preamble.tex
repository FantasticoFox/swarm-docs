

\usepackage{multirow} %for aligning stuff in tables

\usepackage{bbding} %for the smiley face

\usepackage{calc} %calculate spacing \widthof

%tikz stuff
\usepackage{tikzsymbols}
\usepackage{tikz}
\usetikzlibrary{fit,arrows,positioning}

  % Keys to support piece-wise uncovering of elements in TikZ pictures:
  % \node[visible on=<2->](foo){Foo}
  % \node[visible on=<{2,4}>](bar){Bar}   % put braces around comma expressions
  %
  % Internally works by setting opacity=0 when invisible, which has the 
  % adavantage (compared to \node<2->(foo){Foo} that the node is always there, hence
  % always consumes space plus that coordinate (foo) is always available.
  %
  % The actual command that implements the invisibility can be overriden
  % by altering the style invisible. For instance \tikzsset{invisible/.style={opacity=0.2}}
  % would dim the "invisible" parts. Alternatively, the color might be set to white, if the
  % output driver does not support transparencies (e.g., PS) 
  %
  \tikzset{
    invisible/.style={opacity=0},
    visible on/.style={alt={#1{}{invisible}}},
    alt/.code args={<#1>#2#3}{%
      \alt<#1>{\pgfkeysalso{#2}}{\pgfkeysalso{#3}} % \pgfkeysalso doesn't change the path
    },
  }
  \tikzset{
    dimmed/.style={opacity=0.2},
    dimmed on/.style={alt={#1{dimmed}{}}},
    alt/.code args={<#1>#2#3}{%
      \alt<#1>{\pgfkeysalso{#2}}{\pgfkeysalso{#3}} % \pgfkeysalso doesn't change the path
    },
  }
\tikzset{
    %Define standard arrow tip
    >=stealth',
    %Define style for boxes
    peer/.style={
           rectangle,
           rounded corners,
           draw=black, thin,
           text width=3.5em,
           minimum height=2em,
           text centered},
    node/.style={
           rectangle,
           rounded corners,
           draw=black, 
           text width=4.5em,
           minimum height=2em,
           text centered,
           fill={rgb:black,1;white,3}},
    chunk/.style={
           rectangle,
           rounded corners,
           draw=black, 
           text width=2.5em,
           minimum height=1em,
           text centered,
           fill=block title bg},
    % Define arrow style
    point/.style={
           ->,
           thick,
           shorten <=2pt,
           shorten >=2pt,}
every node/.style={align=center}           
}

\newcommand{\wholeslide}[2][]{
\begin{frame}{#1}
% \transboxout<1>[duration=0.5]
\setbeamercolor{bgcolor}{fg=black,bg=white}
\begin{tikzpicture}[overlay, remember picture]
\node[anchor=center] at (current page.center) {
  \begin{beamercolorbox}[center]{bgcolor}
     #2
  \end{beamercolorbox}};
\end{tikzpicture}

\end{frame}
}

\newenvironment<>{varblock}[2][.9\textwidth]{%
  \setlength{\textwidth}{#1}
  \begin{actionenv}#3%
    \def\insertblocktitle{#2}%
    \par%
    \usebeamertemplate{block begin}}
  {\par%
    \usebeamertemplate{block end}%
  \end{actionenv}}

\newlength{\mywidth}
\newcommand{\blockslide}[3][]{
\settowidth{\mywidth}{#3}
\begin{frame}[c]{#1}
\begin{center}
\begin{minipage}{1.1\mywidth}
 \begin{varblock}[1.1\mywidth]{#2}
  #3
 \end{varblock}
\end{minipage}
\end{center}
\end{frame}
}

\mode<presentation>{
\usetheme{Warsaw}\usecolortheme{crane}
\setbeamertemplate{items}[square]
\setbeamertemplate{section in toc}[square]
\setbeamertemplate{subsection in toc}[square]
% \setbeamertemplate{subsection in toc}[subsections numbered]
\setbeamertemplate{subsubsection in toc}[square]
% \setbeamercolor{items}{fg=black,bg=yellow}
\usebeamercolor{block title}
\definecolor{block title bg}{named}{bg}
\setbeamercolor{item projected}{fg=black,bg=block title bg}
\setbeamercolor{itemize item}{fg=block title bg,bg=block title bg}
}




